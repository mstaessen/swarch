\section{Iteration 5: Decomposition of the Scheduler for Anomaly Detection}
\label{add:it5}

\subsection{Step 1: Identify candidate drivers}
\label{add:it5/drivers}

\npar This iteration is driven by:

\begin{itemize}
	\item P1': Timely closure of valves
	\item P2': Anomaly Detection
\end{itemize}

\subsection{Step 2: Choose design concepts}
\label{add:it5/concepts}

\npar The design concepts for this decomposition are completely analogous to the
design concepts of iteration 3 (see section \ref{add:it3/concepts}). 

\subsection{Step 3: Instantiate architectural elements and allocate responsibilities}
\label{add:it5/elements}

\npar Analogous to iteration 3 (see \ref{add:it3}), a Message Router will
enqueue processing jobs with the right priorities in the Queue. The
priorities are determined by the active policy in the Message Router. The
Scheduler will dequeue jobs from the Queue en feed them to the Anomaly Detection
Unit. 

\subsection{Step 4: Define interfaces for instantiated elements}
\label{add:it5/interfaces}

% TODO

\subsection{Step 5: Verify and refine}
\label{add:it5/verification}

% TODO