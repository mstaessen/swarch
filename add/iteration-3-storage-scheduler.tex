\section{Iteration 3: Decomposition of the Storage Scheduler}
\label{add:it3}

\npar In this iteration, the Storage Scheduler is decomposed. 

\subsection{Step 1: Identify candidate drivers}
\label{add:it3/drivers}

\npar The architectural drivers for this iteration were assigned in previous
iterations. The main driver is P2': Timely closure of valves. The main
functionality for this module is for the trames to be stored in a timely matter.

\subsection{Step 2: Choose design concepts}
\label{add:it3/concepts}

\npar Trames of different kinds have different priorities. The scheduler should
treat these frames according to their priorities. For that reason, the scheduler
will keep track of multiple queues, one for each priority level. Four levels
will be needed. 

\begin{description}
	\item[Highest Priority] This queue has the highest priority. If there are jobs
	in this queue, they have to be taken care of first. Alarm trames for gas leaks
	will be initially in this queue. 
	\item[High Priority] This queue handles the normal priority alarm trames. For
	instance, a water leak alarm trame is inserted here. Jobs in this queue can
	only be executed if the highest priority queue is empty. If a job remains in
	this queue for a long time, the job is promoted to the highest priority queue. 
	\item[Normal Priority] This queue handles the low priority alarm trames. For
	instance, a power leak alarm trame is inserted here. Jobs in this queue can
	only be executed if the two above priority queues are empty. If a job remains
	in this queue for a long time, the job is promoted to the high priority queue. 
	\item[Low Priority] This queue has the lowest priority. Jobs in this queue can
	only be processed if the above queues are empty. If a job remains in this queue
	for a long time, the job is promoted to the normal priority queue.
\end{description}

\npar To prevent starvation, the priorities of the individual jobs will be
increased periodically. This can cause a job to move from one queue to another
as indicated in the above list. If a job is in the scheduler for quite a long
time, the job will eventually end up in the highest priority queue and will be
taken care of.

\subsubsection{Design Patterns}
\label{add:it3/patterns}

\paragraph{Collections for States}

\paragraph{Message Router}

\paragraph{Active Object}

\paragraph{}

\subsection{Step 3: Instantiate architectural elements and allocate responsibilities}
\label{add:it3/elements}

\npar 

\subsection{Step 4: Define interfaces for instantiated elements}
\label{add:it3/interfaces}

\subsection{Step 5: Verify and refine}
\label{add:it3/verification}