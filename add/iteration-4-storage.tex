\section{Iteration 4: Decomposition of the Storage Unit}
\label{add:it4}

\subsection{Step 1: Identify candidate drivers}
\label{add:it4/drivers}

\npar This iteration is driven by

\begin{itemize}
	\item Av1: Measurement database failure. The measurements database should be up
	and running 99.9\% of all time. 
	\item Av2': Missing measurements. A failing internal communication component
	should be detected autonomously. 
  	\item P3': Requests to the measurement database. The requests should be
  	handled within a bounded time. 
\end{itemize}

\subsection{Step 2: Choose design concepts}
\label{add:it4/concepts}

\subsubsection{Tactics}
\label{add:it4/tactics}

\paragraph{Availability} 

\npar Av1 states that a 99.9\% uptime must be realized. In order to guarantee
this, three things must be taken into consideration. It should be able to detect
the problem, there should be a way to fix the problem and last but certainly not
least, there should be preventive measures.

\npar As was the case in the previous paragraph, there are many alternatives.
The most frequently used ones are active redundancy, passive redundancy and
spare. The main criterion to select one of these is the downtime of the system.
These downtimes are respectively in the order of miliseconds, seconds and
minutes. An uptime of 99.9\% corresponds to 0.72 hours (= 43.2 minutes) of
permitted downtime on a monthly basis. This number reveals potential problems
when a spare is used because there can be only less the 10 failures a month. For
this reason, use of spares is eliminated.

\npar The difference between active and passive replication is rather subtle.
In an active replication scheme, the system will recover faster then in a
passive replication scheme. However, all replicas should be in a consistent
state. This increases the communication overhead, compared to a passive
replication scheme.

\npar On the other hand, the active replication scheme can also provide load
balancing. If one replica has a higher load than another, the replica with the
lowest load can be used to process the query. For this reason, the active
replication scheme is selected.

\subsubsection{Design Patterns}
\label{add:it4/patterns}

\subsection{Step 3: Instantiate architectural elements and allocate responsibilities}
\label{add:it4/elements}

\subsection{Step 4: Define interfaces for instantiated elements}
\label{add:it4/interfaces}

\subsection{Step 5: Verify and refine}
\label{add:it4/verification}