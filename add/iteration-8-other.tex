\section{Iteration 8: Decomposition of Other}
\label{add:it8}

\subsection{Step 1: Identify candidate drivers}
\label{add:it8/drivers}

\npar In the first iteration there were quite a lot of requirements delegated to
this component. This requires some selection to take place. This selection is
once again based on the priorities of the quality attributes. However since all
the remaning quality attribute scenarios are (split versions from)
modifiability, they can be easily combined. Therefore the drivers for this
iteration are M1', M2 and M3'. 

\npar M3' is linked with the following use cases:

\begin{itemize}
  \item UC14: Request consumption predictions
  \item UC15: Generate invoide
  \item UC16: Mark invoice paid
\end{itemize}

\npar Because of this relation between M3' and the listed use cases their
functionality will all be adressed.

\subsection{Step 2: Choose design concepts}
\label{add:it8/concepts}

\subsubsection{Tactics}
\label{add:it8/tactics}

\paragraph{Modifiability}

\npar For the purpose of modifiability three tactics are selected: anticipate
expected changes, the hiding of information and the use of an intermediary.

\subsubsection{Design Patterns}
\label{add:it8/patterns}

\paragraph{Publisher - Subscriber}

\npar The \emph{Publisher - Subscriber} pattern is in this context particularly useful due to
the inherently unknown parties potentially interested in certain messages. E.g. When
later in time dynamic pricing is introduced it is simply possible for the other
interested parties (customers, HAS, etc.) to subscribe to these events. So this
pattern aids in realizing the anticipate expected changes tactic.

\paragraph{Facade}

\npar To shield the system and it's external actors (i.e. The third party
billing service and utility providers) the \emph{Facade} pattern is interesting.
The pattern suggests a single point of access for both of the aforementioned
actors to access but there is a possibility of bypassing this accesspoint in a
number of sophisticated scenarios. In the context of this decomposition a slight
variation of this pattern is used, this will be explained below. The shielding
functionality of the facade reflects the use of an intermediary (and in a sense
the hiding of information). Notice that there is no explicit design pattern used
for the hiding of information but this is mainly realized in the use of
interfaces for all different components.

\subsection{Step 3: Instantiate architectural elements and allocate responsibilities}
\label{add:it8/elements}

\begin{figure}[H]
	\begin{centering}
		% TODO
		%\includegraphics[width=0.6\textwidth]{figs/decomposition/processing-scheduler/decomposition.pdf}
		\caption{The decomposition of the other component. Iteration
		8}
		\label{fig:add/it8/decomposition}
	\end{centering}
\end{figure}

\npar The full decomposition of this iteration can be viewed in figure
\ref{fig:add/it8/decomposition}. In the subsection below each of the components
will be discussed. 

\subsubsection{Event Channel}

\npar This component is the instantiation of the \emph{Publisher - Subscriber}
pattern's ``Change Propagation Infrastructure''. Its responsibilities are the
handling of subscriptions and publications to certain events. Users of this
channel can be invoke one of the methods of it's interface,
\interface{BillingChannel}. These are discussed in \ref{add:it8/interfaces}.

\subsubsection{Prediction Unit}

\npar The prediction unit is responsible for fetching the consumption
histories of all customers, processing (i.e. running all sorts of
prediction algorithms on the fetched data)and exporting. Therefore it offers a
\interface{Prediction} interface.

\subsubsection{Billing Unit}

\npar The responsibility of this unit is limited to the generation of invoices.
Ths includes determining that an invoice should be constructed, fetching all the
relevant consumption data and constructing the invoice. The constructed invoice
is then sent towards the billing communication unit through the
\interface{BillingReceive} interface. One more task is assigned to this
component, namely the marking of paid invoices. To retrieve all data the
billing unit needs, first of all it has to have access through the internal
ReMeS customer database (through the \interface{CustomerProfile} interface).
Secondly, it needs to be able to contact the UIS. This is done through the
\interface{CustomerInformation}.

\subsubsection{UIS Communication Unit}

\npar This unit is the first of two units responsible for communication with
external actors in respect to billing and prediction. In section
\ref{add:it8/patterns} we discussed the usage of a slight variant of the
\emph{Facade} pattern. The deviation of the standard pattern lies firstly in the
usage of two communication units instead of one single point of access.
Secondoly there is no bypassing of this communication (not even in a limited
number of scenarios).

\npar This component uses three interfaces. The first one, \interface{UIS}, is
used for all information needed by ReMeS that is stored by the utility
company. The second one, \interface{Prediction}, is used to acquire prediction
statistics for a certain utility company. Besides using interfaces the UIS
communication unit off course offers interfaces itself. It offers two
interfaces, one towards the billing unit (\interface{CustomerInformation}) and
one towards the UIS (\interface{UISCommunication}). The former is used for the
retrieval of customer data that is kept by the UIS. The latter one serves as a
provider of different services for the UIS.

\subsubsection{Billing Communication Unit}

\npar The responsibility of this unit is the handling of all information between
ReMeS and external actors regarding billing. It offers two interfaces:
\interface{BillingSend} and \interface{BillingCommunication}. The first
interface is provided to allow invoices to be sent towards the third party
billing services after it has been constructed. The latter interface is
provided for all notification that need to be sent towards ReMeS concerning the
payment of invoices. 

\npar The billing communication unit also uses functionality which is reflected
in the usage of the \interface{BillingReceive} and \interface{TPBS} interfaces.
The \interface{BillingReceive} simply has as goal to sent the notification of
payment towards the billing unit. The other interface takes care of the
communication towards the third party billing service.

\subsection{Step 4: Define interfaces for instantiated elements}
\label{add:it8/interfaces}

\paragraph{}

\subsection{Step 5: Verify and refine}
\label{add:it8/verification}