\subsubsection{T05 -- Information Disclosure of transmitted personal information}
\label{threats:t05}

\paragraph{Summary}

\npar The misactor gains access to the data flow that contains personal
information of a user. This personal information can be various things, e.g.
credentials, detailed consumption information, etc.

\paragraph{Primary mis-actor}

\npar Skilled insider (e.g. admin)

\paragraph{Basic path}
\begin{enumerate}
	\item[bf1.] The misactor gains access to the dataflow between the Consumer
	Portal and the user data store.
    \item[bf2.] The misactor intercepts personal information.
\end{enumerate}

\paragraph{Consequence}

\npar The misactor now has access to the user's information and can possibly
spoof the user when his or her credentials were stolen. When other personal
information is retrieved the misactor can sell this information to companies.

\paragraph{Reference to threat tree node(s)}

ID\_df4, ID\_df7

\paragraph{Parent threat tree(s)}

ID\_df

\paragraph{DFD element(s)}

This applies to all internal dataflows. More specific, to all dataflows which
were collapsed into the ``1.1 Incoming Communication Component - 1.11 Process
Data'' dataflow.

\paragraph{Remarks:}
	\begin{enumerate}
         \item[r1.] The personal information which was mentioned throughout the
         misuse case is very broad and is a generalisation of all data which is
         important the user. This includes: credentials, consumption
         information, contact details, etc.
         \item[r2.] Side channel attacks are not considered (see assumption 6).
         \item[r3.] This threat is possible because interal dataflows are
         possibly not encrypted.
    \end{enumerate}
