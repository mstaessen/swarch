\subsubsection{T02 - Linking measurementsdata to user data}
\label{threats:t02}
%TODO cascading preconditions voor de threats
\paragraph{Summary}

\npar An insider with access to both data stores (i.e. the measurements and
information data store and the user data store) is able to link data from both
databases (and commercialise this knowledge by e.g. selling it to other
companies).

\paragraph{Primary mis-actor}

\npar unskilled insider with access to both data stores.

\paragraph{Basic path}

\begin{enumerate}
	\item[bf1.] The misactor retrieves information from both the measurement and
	information data store and the user data store.
    \item[bf2.] The misactor subsequently links both sets of data (e.g. based
    on a shared foreign key)
\end{enumerate}

\paragraph{Consequence}

\npar The combined set of data contains (possibly sensitive) personal
identifiable information and especially poses a privacy threat when the misactor
commercialises the information.

\paragraph{Reference to threat tree node(s)}

L\_ds3

\paragraph{Parent threat tree(s)}

L\_ds

\paragraph{DFD element(s)}

\begin{itemize}
  \item 1.8 Measurement and information data
  \item 1.9 User data
  \item 1.10 Alarm configuration data
\end{itemize}

\paragraph{Remarks}
	\begin{enumerate}
         \item[r1.] Altough this threat describes the linking of measurements
         and information data to user data, it also applies to all other
         combinations of data stores.
         \item[r2.] Because of assumption %TODO ref
		 , the misactor has access to several (strong) data mining technique and hence
		 L\_ds2 is fulfilled
         \item[r3.] Node L\_ds1 is applicable due to assumption %TODO ref.
         \item[r4.] L\_ds3 is mentioned because this is exactly what this threat
         describes, namely the linking of different databases.
         \item[r5.] This threat can be used a precondition for the identiability
         threat of data stores (%TODO ref naar T03)
    \end{enumerate}
