\subsubsection{T01 -- Linking measurements data}
\label{threats:t01}

\paragraph{Summary}

\npar A researcher or other insider with malicious intents links measurement,
user or alarm configuration data.

\paragraph{Primary mis-actor}

\npar unskilled insider (authenticated user, e.g. researcher)

\paragraph{Basic path}
\begin{enumerate}
	\item[bf1.] The misactor performs a set of targeted queries on the
	measurements data and retrieves very detailed results.
    \item[bf2.] The misactor links the results of the queries together (e.g.
    based on the regio where the measurements are from).
\end{enumerate}

\paragraph{Consequence}

\npar By applying these data mining algorithms on the query results the misactor
has more knowledge of information about ReMeS or users than desirable. 

\paragraph{Reference to threat tree node(s)} 

L\_ds1, L\_ds2


\paragraph{Parent threat tree(s)}

L\_ds

\paragraph{DFD element(s)}

1.8 Measurement and information data, 1.9 User data, 1.10 Alarm configuration
data

\paragraph{Remarks}

	\begin{enumerate}
         \item[r1.] Altough this threat mainly describes the measurements
         and information data, it also applies to all other data stores, namely
         user and alarm configuration data.
         \item[r2.] Because of assumption 5, the misactor has access to several
         (strong) data mining technique and hence L\_ds2 is fulfilled.
         \item[r3.] Node L\_ds1 is applicable due to assumption 4.
         \item[r4.] This threat serves as precondition for the
         identiability threat of data stores (T03).
    \end{enumerate}
