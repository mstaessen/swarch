\subsubsection{T06 -- Linkability of bills sent to the Third Party Billing Web Service}

\paragraph{Summary} The misactor links several bills to the same consumer and
creates a profile of this consumer. 

\paragraph{Primary mis-actor} unskilled insider (Third Party Billing Web
Service)

\paragraph{Basic path}
\begin{enumerate}
	\item[bf1.] New bills are sent to the third party billing web service.
	\item[bf2.] The misactor intercepts the data flow.
	\item[bf3.] The misactor links several bills to the same consumer.
\end{enumerate}

\paragraph{Consequence} The misactor can build a profile of the consumer

\paragraph{Reference to threat tree node(s)} L\_df3, L\_df10

\paragraph{Parent threat tree(s)} L\_df

\paragraph{DFD element(s)} data flow from 1.7 Billing to 5. Third Party Billing
Web Service

\paragraph{Remarks}
\begin{enumerate}
	\item[r1.] The misactor is the receiver, which justifies this threat.
	\item[r2.] The data flow is encrypted (see assumption 15), which will prevent
	information disclosure to outsiders.
	\item[r3.] Different bills are linked based on the consumer's ID (L\_df10). 
	\item[r4.] The right branch of the three (L\_df4 and children) is not
	considered because it is not the sender whose identity should be protected, but the
	consumer, who is not directly a part of the data flow.
\end{enumerate}
