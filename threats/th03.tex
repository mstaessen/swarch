\subsubsection{T03 -- Identifying a consumer from his user data}
\label{threats:t03}

\paragraph{Summary}

\npar A researcher with malicious intent identifies a consumer in a set of user
(or measurement or alarm configuration) data.

\paragraph{Primary mis-actor}

\npar unskilled insider.

\paragraph{Basic path}
\begin{enumerate}
	\item[bf1.] The misactor performs a set of targeted queries on the user
	(or measurements or alarm configuration) data store and retrieves very detailed
	results.
    \item[bf2.] The misactor can extract the identity of the customer from each
    individual query result because of strong data mining techniques or he first
    links several results to each other (see threats T01 and T02) which provide
    him with identiable information.
\end{enumerate}

\paragraph{Consequence}

\npar The misactor gains access to the consumer's identity although this should
have remained secret.

\paragraph{Reference to threat tree node(s)}

I\_ds1, I\_ds2

\paragraph{Parent threat tree(s)}

I\_ds

\paragraph{DFD element(s)}

1.8 Measurement and information data store, 1.9 User data store, 1.10 Alarm
configuration data store

\paragraph{Remarks}
	\begin{enumerate}
         \item[r1.] Threats T01 and T02 are (part of) the preconditions for
         this threat.
         \item[r2.] Because of assumption 5, the misactor has access to several
         (strong) data mining technique and hence I\_ds2 is fulfilled.
         \item[r3.] Node I\_ds1 is applicable due to assumption 4.
         \item[r4.] Although this threat is elaborated for user data it is as
         well applicable for the other datastores.
    \end{enumerate}
