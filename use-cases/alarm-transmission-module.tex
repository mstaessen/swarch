\begin{description}
	\item[Primary actor] Remote monitoring module
	\item[Interested parties] Customer, Call Center, Utilty providing company,
	ReMeS, Emergency services
	\item[Preconditions] \ 
	\begin{itemize}
	  	\item An anomaly detection algorithm triggered the alarm.
		\item The remote module is installed and active.
		\item The customer needs to have an alarm configuration.
	\end{itemize}
	\item[Postconditions] \ 
	\begin{itemize}
		\item The event is logged.
		\item The consumption profile is updated.
	\end{itemize}
	\item[Normal flow] \ 
	\begin{enumerate}
	  	% 1
	  	\item The remote monitoring module prepares an alarm trame.
	  	% 2
	  	\item The remote monitoring module sends the alarm trame to the configured
	  	data center. 
	  	% 3 
	  	\item The data center collects the alarm trame.
		% 4
	  	\item The data center processes the alarm trame.
		% 5
	  	\item The data center notifies the customer. Include use case
	  	\ref{uc-alarm-transmission-remes} (``Alarm transmission: ReMeS'').
	\end{enumerate}
	\item[Alternate flow] \ 
	\begin{description}
		\item[4a] The trame is coming from a remote gas monitoring module.
			\begin{enumerate}
				\item ReMeS sends a shutdown command to the remote valve.
				\item The remote valve shuts down.
				\item The remote valve acknowledges to ReMeS if the shutdown was successful.
			\end{enumerate}
		\item[5a] The data center recognizes this alarm trame as a false alarm.
		\begin{enumerate}
			\item ReMeS records a false alarm.
			\item The use case ends here. 
		\end{enumerate}
	\end{description}
	\item[Exception flow] \ 
	\begin{description}
		\item[2a] The trame could not been sent. 
		\begin{enumerate}
			\item The remote module waits for 3 seconds.
			\item Return to step 2 of the normal flow.
		\end{enumerate}
	\end{description}
\end{description}