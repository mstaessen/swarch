\begin{description}
	\item[Primary actor] Remote monitoring module
	\item[Interested parties] Customer, ReMeS, Utility providing company 
	\item[Preconditions] \ 
	\begin{itemize}
		\item The remote module is installed.
		\item The remote module is activated.
	\end{itemize}
	\item[Postconditions] \ 
	\begin{itemize}
		\item The data center has collected a trame.
	\end{itemize}
	\item[Normal flow] \ 
	\begin{enumerate}
	  	% 1
		\item The remote module prepares a data trame with all the information (date
		and time, battery level, counter values, \ldots).
		% 2
		\item The remote module sends the trame to the configured phone number by SMS.
		% 3
		\item The communication unit collects the trame.
		% 4 
		\item The communication unit creates a corresponding event.
		% 5
		\item The storage unit stores the event. 
		% 6
		\item The anomaly detection unit processes the trame.
		% 7
		\item The analytics unit processes the trame.
	\end{enumerate}
	\item[Alternate flow] \ 
	\begin{description}
		\item[1a] The battery is low.
		\begin{enumerate}
			\item Include Use Case \ref{uc:send-low-battery-alarm} (``Send low battery
			alarm'').
		\end{enumerate}
		\item[2a] The remote module uses the internet to send trames.
		\begin{enumerate}
			\item The remote module sends the trame to the configured IP address using
			the configured WiFi connection.
			\item Return to step 3 in the normal flow.
		\end{enumerate}
		\item[6a] An anomaly is detected.
		\begin{enumerate}
			\item Include use case \ref{uc:send-alarm} (``Send alarm'').
			\item Return to step 7 in the normal flow.  
		\end{enumerate} 
	\end{description}
	\item[Exception flow] \ 
	\begin{description}
		\item[2a] The trame cannot be sent (timeout or wrong response code).
		\begin{enumerate}
			\item The remote module waits for 5 minutes;
			\item Return to step 1 in the normal flow.  
		\end{enumerate}
	\end{description}
\end{description}