\section{Eliciting threats}
\label{sec:eliciting-threats}

\subsection{Assumptions}

\npar This sections includes all general and specific assumptions made to
influence the elicitation process.

\begin{enumerate}

  %General
  \item Non-compliance will be modelled as a single threat to the whole system
  instead of differentiating between all \textsc{dfd} elements.
  \item Non-repudiation does not form a threat to the system, in fact it is a
  desirable aspect of the system. With no non-repudiation it is for
  example possible for a client to deny that he consumed the measured amounts
  of the different utilities.
  \item For datastores, dataflows and processes, detectability is not taken into
  account as a privacy threat. It is only important that the data itself is not
  leaked instead of the metadata. The core business of ReMeS after all involves
  measurements, modules, etc. Knowing that this information is present in the
  system is no big deal.
  \item The databases have weak access control since insiders can easily read
  the databases.
  \item Misactors have access to strong data mining techniques, since
  there is an enormous supply of optimized machine learing libraries available
  on the Internet.
  \item Side channel attacks are not considered to occur since they take a lot
  of analysis and the extracted information is not in correspondence of the
  effort.
   
  %Data stores
  \item All internal data stores are kept separate even though they are the
  possible target of the same threats (as seen in table %todo refentie ).
  This is done to stress the fact that the different databases could be linked
  together, this is handled in a separate misuse case.
  \item Linkability is a threat to the data stores because for example a
  researcher with malicious intents could link several data and commercialise
  this information.
  \item Identifiability forms as well a threat for data stores because it is
  impossible to omit all identification details out of the data. 
  \item Information Disclosure for data stores is logic because they contain all
  the data and this data needs to be protected at all times.
  \item Datastores are sufficiently protected so extra-monitor and bad-storage
  management are not possible.
  
  % data flows
  \item All internal dataflows (i.e. dataflows between two internal processes
  and dataflows between an internal process and an internal data store) are only
  susceptible to insider threats, as we consider the back-end sufficiently
  protected against outsider threats. All these dataflows will therefore be
  combined into a single dataflow and only this dataflow will be examined.
  \item All bidirectional flows are combined into a single flow as there is no
  difference between the two directions. 
  \item All Portal - Entity dataflows are combined into a single flow because
  there is no difference between regarding privacy. %TODO is user data niet veel
  % gevoeliger dan bvb registeren van devices zoals (cf. operator portal)
  \item Detectability and Information Disclosure are not considered as a threats
  for external dataflows (i.e. flows between an entity and a process) as we
  assume that those flows are encrypted (e.g. with the SSL protocol).
  \item Because the flows are assumed to be encrypted, all external entities
  will need to be able to execute these encryption algorithms themselves. This
  might form a problem for remote modules as their resources are limited.
  However, there are two reasons why this will not cause problems. First of all
  the modules can run leak detection algorithms so running the encryption
  algorithms in addition to the leak detection ones will cause performance
  issues. Second, modules which communicate through WiFi need this encryption
  functionality anyway to communicate.
  \item It is possible that there is Information Disclosure on the Outgoing
  Communication Component - Consumer dataflow when the alarm recipients are
  wrong (e.g. an email adres belonging to somebody else). However, it is the
  responsibility of the consumer to fill in these contact details correctly.
  Therefore this kind of Identification Disclosure is left out of
  consideration.
  \item Linkability and identiability are a threat for only a specific
  number of external dataflows, namely the one which involves interaction with a
  third party (i.e. billing service). The flow between UIS Communication
  Component and the UIS Web Service is not applicable for these theats
  because the contents of the flow are not personal. It is important to remark
  that the predictions will therefore not contain any customer profiles like
  for example customer X will consume an amount Y of utility Z. If the dataflow
  did contain such profiles, the contents would be personal.
  \item For the flow from and towards consumers and remote modules linkability
  and identiability threats do not apply. This is logical considering that there
  need to be unique identifiers to contact both of these entities.

  %Processes.
  \item All internal processes are only susceptible to insider threats, as we
  consider the back-end sufficiently protected against outsider threats. All
  these processes will therefore be combined into a single process and only
  this process will be examined. 
  \item Linkability and identifiability are not considerd as threats for
  processes because knowing that two actions belong to the same user does not
  violate that user's privacy. His or her privacy is only violated when the
  content of the action is revealed.
  \item Information Disclosure is still a threat that needs to be taken into
  account.
  
  % Entities
  \item Identifiability and linkability are no threat to the system as each
  of the entities has a unique identifier and there is no need to conceal this
  identity. 
  \item Content unawareness is only applicable to Customers as they are the only
  actors that enter actual (personal) data in the system. Other entities, like
  for example researchers only retrieve data from the system, the ReMeS
  personnel only retrieves statistics and registers new devices. Remote modules
  send measurements. The third party billing web service only registers whether
  bills are paid. Finally the UIS Communication component updates prices.
  Hence, all of these entities are not susceptible to the content unwareness
  threat.
\end{enumerate}

\subsection{Threats}

\subsubsection{T01 -- Linking measurements data}
\label{threats:t01}

\paragraph{Summary}

\npar A researcher or other insider with malicious intents links measurement,
user or alarm configuration data.

\paragraph{Primary mis-actor}

\npar unskilled insider (authenticated user, e.g. researcher)

\paragraph{Basic path}
\begin{enumerate}
	\item[bf1.] The misactor performs a set of targeted queries on the
	measurements data and retrieves very detailed results.
    \item[bf2.] The misactor links the results of the queries together (e.g.
    based on the regio where the measurements are from).
\end{enumerate}

\paragraph{Consequence}

\npar By applying these data mining algorithms on the query results the misactor
has more knowledge of information about ReMeS or users than desirable. 

\paragraph{Reference to threat tree node(s)} 

L\_ds1, L\_ds2


\paragraph{Parent threat tree(s)}

L\_ds

\paragraph{DFD element(s)}

1.8 Measurement and information data, 1.9 User data, 1.10 Alarm configuration
data

\paragraph{Remarks}

	\begin{enumerate}
         \item[r1.] Altough this threat mainly describes the measurements
         and information data, it also applies to all other data stores, namely
         user and alarm configuration data.
         \item[r2.] Because of assumption 5, the misactor has access to several
         (strong) data mining techniques and hence L\_ds2 is fulfilled.
         \item[r3.] Node L\_ds1 is applicable due to assumption 4.
         \item[r4.] This threat serves as precondition for the
         identiability threat of data stores (T03).
    \end{enumerate}

\subsubsection{T02 -- Linking measurements data to user data}
\label{threats:t02}

\paragraph{Summary}

\npar An insider with access to both data stores (i.e. the measurements and
information data store and the user data store) is able to link data from both
databases (and commercialise this knowledge by e.g. selling it to other
companies).

\paragraph{Primary mis-actor}

\npar unskilled insider with access to both data stores.

\paragraph{Basic path}

\begin{enumerate}
	\item[bf1.] The misactor retrieves information from both the measurements and
	information data store and the user data store.
    \item[bf2.] The misactor subsequently links both sets of data (e.g. based
    on a shared foreign key)
\end{enumerate}

\paragraph{Consequence}

\npar The combined set of data contains (possibly sensitive) personal
identifiable information and especially poses a privacy threat when the misactor
commercialises the information.

\paragraph{Reference to threat tree node(s)}

L\_ds3, L\_ds2, L\_ds1

\paragraph{Parent threat tree(s)}

L\_ds

\paragraph{DFD element(s)}

1.8 Measurement and information data, 1.9 User data, 1.10 Alarm configuration
data

\paragraph{Remarks}
	\begin{enumerate}
         \item[r1.] Altough this threat describes the linking of measurements
         and information data to user data, it also applies to all other
         combinations of data stores.
         \item[r2.] Because of assumption 5, the misactor has access to several
         (strong) data mining techniques and hence L\_ds2 is fulfilled.
         \item[r3.] Node L\_ds1 is applicable due to assumption 4.
         \item[r4.] L\_ds3 is mentioned because this is exactly what this threat
         describes, namely the linking of different databases.
         \item[r5.] This threat can be used a precondition for the identiability
         threat of data stores (T03).
    \end{enumerate}

\subsubsection{T03 - identifying a customer from his user data}
\label{threats:t03}

\paragraph{Summary}

\npar A researcher with malicious intent identifies a customer in a set of user
(or measurement or alarm configuration) data.

\paragraph{Primary mis-actor}

\npar unskilled insider.

\paragraph{Basic path*:}
\begin{enumerate}
	\item[bf1.] The misactor performs a set of targeted queries on the user
	(or measurement or alarm configuration) data store and retrieves very detailed
	results.
    \item[bf2.] The misactor can extract the identity of the customer from each
    individual query result because of strong data mining techniques or he first
    links several results to each other (see threats \ref{threats:t01} and
    \ref{threats:t02}) which provide him with identiable information.
    \item[bf3.]
\end{enumerate}

\paragraph{Consequence}

\npar The misactor gains access to the customer's identity although this should
have remained secret.

\paragraph{Reference to threat tree node(s)}

\begin{itemize}
  \item I\_ds1
  \item I\_ds2
\end{itemize}

\paragraph{Parent threat tree(s)}

\begin{itemize}
  \item I\_ds
\end{itemize}

\paragraph{DFD element(s)}

\begin{itemize}
  \item 1.8 Measurement and information data store
  \item 1.9 User data store
  \item 1.10 Alarm configuration data store
\end{itemize}

\paragraph{Remarks}
	\begin{enumerate}
         \item[r1.] Threats \ref{threats:t01} and \ref{threats:t02} are
		(part of) the preconditions for this threat.
         \item[r2.] Because of assumption %TODO ref
		 , the misactor has access to several (strong) data mining technique and hence
		 I\_ds2 is fulfilled
         \item[r3.] Node I\_ds1 is applicable due to assumption %TODO ref.
         \item[r4.] Although this threat is elaborated for user data it is as
         well applicable for the other datastores.
    \end{enumerate}

\subsubsection{T04 - Information disclosure of data}
%TODO: staat geen relevant voorbeeld van in, in de het example report ?

\paragraph{Summary}
\paragraph{Primary mis-actor}
\paragraph{Basic path}
\begin{enumerate}
	\item[bf1.]{}
    \item[bf2.]{}
    \item[bf3.]{}
\end{enumerate}
\paragraph{Consequence}
\paragraph{Reference to threat tree node(s)}
\paragraph{Parent threat tree(s)}
\paragraph{DFD element(s)}
\paragraph{Remarks:}
	\begin{enumerate}
         \item[r1.]
         \item[r2.]
    \end{enumerate}

\subsubsection{T05 -- Information Disclosure of transmitted personal information}
\label{threats:t05}

\paragraph{Summary}

\npar The misactor gains access to the data flow that contains personal
information of a user. This personal information is quite broad amongst it are
e.g. credentials, detailed consumption information, etc.

\paragraph{Primary mis-actor}

\npar Skilled insider (e.g. admin)

\paragraph{Basic path}
\begin{enumerate}
	\item[bf1.] The misactor gains access to the dataflow between the Consumer
	Portal and the user data store.
    \item[bf2.] The misactor intercepts personal information.
\end{enumerate}

\paragraph{Consequence}

\npar The misactor now has access to the user's information and can possibly
spoof the user when his or her credentials were stolen. When other personal
information is retrieved the misactor can sell this information to companies.

\paragraph{Reference to threat tree node(s)}

ID\_df4, ID\_df7

\paragraph{Parent threat tree(s)}

ID\_df

\paragraph{DFD element(s)}

This applies to all internal dataflows. More specific, to all dataflows which
were collapsed into the ``1.1 Incoming Communication Component - 1.11 Process
Data'' dataflow.

\paragraph{Remarks:}
	\begin{enumerate}
         \item[r1.] The personal information which was mentioned throughout the
         misuse case is very broad and is a generalisation of all data which is
         important the user. This includes: credentials, consumption
         information, contact details, etc.
         \item[r2.] Side channel attacks are not considered (see assumption
         %TODO ref)
         \item[r3.] This threat is possible because interal dataflows are
         possibly not encrypted.
    \end{enumerate}

\subsubsection{T06 -- Linkability of bills sent to the Third Party Billing Web Service}

\paragraph{Summary} The misactor links several bills to the same consumer and
creates a profile of this consumer. 

\paragraph{Primary mis-actor} unskilled insider (Third Party Billing Web
Service)

\paragraph{Basic path}
\begin{enumerate}
	\item[bf1.] New bills are sent to the third party billing web service
	\item[bf2.] The misactor intercepts the data flow
	\item[bf3.] The misactor links several bills to the same consumer
\end{enumerate}

\paragraph{Consequence} The misactor can build a profile of the consumer

\paragraph{Reference to threat tree node(s)} L\_df3, L\_df10

\paragraph{Parent threat tree(s)} L\_df

\paragraph{DFD element(s)} data flow from 1.7 Billing to 5. Third Party Billing
Web Service

\paragraph{Remarks}
\begin{enumerate}
	\item[r1.] The misactor is the receiver, which justifies this threat.
	\item[r2.] The data flow is encrypted (see assumption ??), which will prevent
	information disclosure to outsiders. % TODO: assumptie encryptie
	\item[r3.] Different bills are linked based on the consumer's ID (L\_df10). 
	\item[r4.] The right branch of the three (L\_df4 and children) is not
	considered because it is not the sender whose identity should be protected, but the
	consumer, who is not directly a part of the data flow.
\end{enumerate}

\subsubsection{T07 -- Identifiability of bills sent to the Third Party Billing
Web Service}

\paragraph{Summary} The misactor extracts the consumer's identity from the sent
bills and links it to other bills.

\paragraph{Primary mis-actor} unskilled insider (Third Party Billing Web
Service)

\paragraph{Basic path}
\begin{enumerate}
	\item[bf1.] New bills are sent to the third party billing web service
	\item[bf2.] The misactor intercepts the data flow
\end{enumerate}

\paragraph{Consequence} The misactor can build a financial and consumption
profile of the consumer. 

\paragraph{Reference to threat tree node(s)} I\_df3, I\_df10

\paragraph{Parent threat tree(s)} I\_df

\paragraph{DFD element(s)} data flow from 1.7 Billing to 5. Third Party Billing
Web Service

\paragraph{Remarks}
\begin{enumerate}
	\item[r1.] The misactor is the receiver, which justifies this threat.
	\item[r2.] The data flow is encrypted (see assumption ??), which will prevent
	information disclosure to outsiders.
	% TODO: assumptie encryptie
	\item[r3.] The consumer is identifiable, based on the identifier on the bill
	(I\_df10).
	\item[r4.] The right branch of the three (I\_df4 and children) is not
	considered because it is not the sender whose identity should be protected, but the
	consumer, who is not directly a part of the data flow.
\end{enumerate}

\subsubsection{T08 -- Information disclosure internal process}

\paragraph{Summary} The misactor gains control to one of the internal processes.

\paragraph{Primary mis-actor} authorized insider

\paragraph{Basic path}
\begin{enumerate}
	\item[bf1.] The misactor has the required privileges to access the process.
	\item[bf2.] The misactor uses these privileges to access information outside
	the scope of his job.
\end{enumerate}

\paragraph{Consequence} The misactor has access to (possibly sensitive)
personal information.

\paragraph{Reference to threat tree node(s)} ID\_p

\paragraph{Parent threat tree(s)} ID\_p

\paragraph{DFD element(s)} all internal processes

\paragraph{Remarks}
\begin{enumerate}
     \item[r1.] Only authorized insiders are considered because it is assumed
     that the system is sufficiently protected from outsiders as stated by
     assumption 20.
\end{enumerate}

\subsubsection{T09 -- Content inaccuracy} 

\paragraph{Summary} The user fails to update contact information. 

\paragraph{Primary mis-actor} Management

\paragraph{Basic path}
\begin{enumerate}
	\item[bf1.] The management fails to inform the consumer about the importance of
	up-to-date information (like for example, alarm notification addresses).
    \item[bf2.] The consumer fails to update old information.
\end{enumerate}

\paragraph{Consequence} The user might become unaware of a valve shutdown if his
alarm notification addresses are not up-to-date.

\paragraph{Reference to threat tree node(s)} U\_3, U\_4

\paragraph{Parent threat tree(s)} U

\paragraph{DFD element(s)} 3. Consumer

\paragraph{Remarks}
\begin{enumerate}
     \item[r1.] This threat only applies to consumers as stated by assumption
     24.
     \item[r2.] This threat is also applicable for other contact information.
     ReMeS should always be able to contact the customer.
\end{enumerate}

\subsubsection{T10 -- Non-compliance}

\paragraph{Summary} ReMeS does not process data in compliance with legislation
and/or policies (ReMeS policy, Utility Company Policy, \ldots).

\paragraph{Primary mis-actor} insider (people with access to the data)

\paragraph{Basic path}
\begin{enumerate}
	\item[bf1.] The misactor fails to comply with the ReMeS policy and/or
	privacy legislation.
\end{enumerate}

\paragraph{Consequence} The user's information is shared without his knowledge.
If detected, lawsuits will probably follow and ruin the reputation of the
system.

\paragraph{Reference to threat tree node(s)} PN\_2

\paragraph{Parent threat tree(s)} PN

\paragraph{DFD element(s)} all (except entities)

\paragraph{Remarks}
\begin{enumerate}
     \item[r1.] This threat applies to the system as a whole, no individual DFD
     element is specifically targetted (see assumption 1).
	\item[r3.] A specific non-compliance threat concerning consents is
     described in T11.
\end{enumerate}
\subsubsection{T11 -- Missing consumer consents}

\paragraph{Summary} The system did not ask the consumer's permission to share
part of his (pseudonymized) measurements information with researchers, other
consumers and utility companies.

\paragraph{Primary mis-actor} Management

\paragraph{Basic path}
\begin{enumerate}
	\item[bf1.] The management fails to require consumer's consents to be included
	in the data flow.
	\item[bf2.] The consumer is unable to state his preferences concerning sharing
	measurements data.
\end{enumerate}

\paragraph{Consequence} The consumer's measurements data will be shared with
others against his will. 

\paragraph{Reference to threat tree node(s)} PN\_3

\paragraph{Parent threat tree(s)} PN

\paragraph{DFD element(s)} entire system (except entities)

\paragraph{Remarks}
\begin{enumerate}
     \item[r1.] This threat applies to the entire system (see assumption 1).
     \item[r2.] A general threat which corresponds to general non-compliance is
     described in T10.
\end{enumerate}

