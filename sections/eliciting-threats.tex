\section{Eliciting threats}
\label{sec:eliciting-threats}

\subsection{Assumptions}

\npar This sections includes all general and specific assumptions made to
influence the elicitation process.

\begin{enumerate}

  %General
  \item Non-compliance will be modelled as a single threat to the whole system
  instead of differentiating between all \textsc{dfd} elements.
  \item Non-repudiation does not form a threat to the system, in fact it is a
  desirable aspect of the system. With no non-repudiation it is for
  example possible for a client to deny that he consumed the measured amounts
  of the different utilities.
  \item For datastores, dataflows and processes, detectability is not taken into
  account as a privacy threat. It is only important that the data itself is not
  leaked instead of the metadata. The core business of ReMeS after all involves
  measurements, modules, etc. Knowing that this information is present in the
  system is no big deal.
  
  %Data stores
  \item All internal data stores are only susceptible to insider threats, as we
  consider the back-end sufficiently protected against outsider threats. All
  these dataflows will therefore be combined into a single data store and only
  this data store will be examined.
  \item Linkability is a threat to the data stores because for example a
  researcher with malicious intents could link several data and commercialise
  this information.
  \item Identifiability forms as well a threat for data stores because it is
  impossible to omit all identification details out of the data. 
  \item Information Disclosure for data stores is off course applicable to the
  data stores because the contain all the data and this data needs to be
  protected at all times.
  
  % data flows
  \item All internal dataflows (i.e. dataflows between two internal processes
  and dataflows between an internal process and an internal data store) are only
  susceptible to insider threats, as we consider the back-end sufficiently
  protected against outsider threats. All these dataflows will therefore be
  combined into a single dataflow and only this dataflow will be examined.
  \item All bidirectional flows are combined into a single flow as there is no
  difference between the two directions. Notice that fundamentally different
  flows (e.g. consumer portal - consumer and operator portal - ReMeS personnel)
  are not simplified into one flow because the information sent over the flow
  is different.
  \item Detectability and Information Disclosure are not considered as a threats
  for external dataflows (i.e. flows between an entity and a process) as we
  assume that those flows are encrypted (e.g. with the SSL protocol).
  \item Because the flows are assumed to be encrypted, all external entities
  will need to be able to execute these encryption algorithms themselves. This
  might form a problem for remote modules as their resources are limited.
  However, there are two reasons why this will not cause problems. First of all
  the modules can run leak detection algorithms so running the encryption
  algorithms in addition to the leak detection ones will cause performance
  issues. Second, modules which communicate through WiFi need this encryption
  functionality anyway to communicate.
  \item It is possible that there is Information Disclosure on the Outgoing
  Communication Component - Consumer dataflow when the alarm recipients are
  wrong (e.g. an email adres belonging to somebody else). However, it is the
  responsibility of the consumer to fill in these contact details correctly.
  Therefore this kind of Identification Disclosure is left out of consideration.
  \item Linkability and identiability are a threat for external dataflows as a
  for example a researcher with malicious intents could try to link a number of
  dataflows and hence gather information about certain entities.

  %Processes.
  \item All internal processes are only susceptible to insider threats, as we
  consider the back-end sufficiently protected against outsider threats. All
  these processes will therefore be combined into a single process and only
  this process will be examined. 
  \item Linkability and identifiability are not considerd as threats for
  processes because knowing that two actions belong to the same user does not
  violate that user's privacy. His or her privacy is only violated when the
  content of the action is revealed.
  \item Information Disclosure is still a threat that needs to be taken into
  account.
  
  % Entities
  \item Identifiability and linkability are no threat to the system as each
  of the entities has a unique identifier and there is no need to conceal this
  identity. 
  \item Content unawareness is only applicable to Customers as they are the only
  actors that enter actual (personal) data in the system. Other entities, like
  for example researchers only retrieve data from the system, the ReMeS
  personnel only retrieves statistics and registers new devices. Remote modules
  send measurements. The third party billing web service only registers whether
  bills are paid. Finally the UIS Communication component updates prices.
  Hence, all of these entities are not susceptible to the content unwareness
  threat.
\end{enumerate}

\subsection{Threats}

\subsubsection{ID + title*}
% the attributes noted with an asterix (*) are obligated
\paragraph{Summary*:}
\paragraph{Primary mis-actor*:}
\paragraph{Basic path*:}
\begin{enumerate}
	\item[bf1.]{}
    \item[bf2.]{}
    \item[bf3.]{}
\end{enumerate}
\paragraph{Consequence*:}
\paragraph{Reference to threat tree node(s)*:}
\paragraph{Parent threat tree(s)*:}
\paragraph{DFD element(s)*:}
\paragraph{Remarks:}
	\begin{enumerate}
         \item[r1.]
         \item[r2.]
    \end{enumerate}
