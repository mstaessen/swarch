\section{Prioritization}
\label{sec:prioritization}

\npar This section lists all the threats (ID and title) according to their
priority. This ordering is based on the risk of the threat which depends on one
side on the likelihood and on the other side on the impact. The threats are
divided into high, medium and low priority ones. Each of these categories is
subsequently ordered according to the risk of the threats in it. In addition to
this ordering, a brief explanation is given why the threats are ordered into
that specific order.

\subsection{High Priority}

\ %Lay-out

\begin{itemize}
  \item T04 -- Information disclosure of user data
  \item T05 -- Information disclosure of transmitted personal information
  \item T03 -- Identifying a customer from his user data  
\end{itemize}

\npar Information disclosure of data is the most important threat as it violates
the customer's privacy the most (the customer enters his personal information
under the assumption that this information is kept confidential).

\npar When a data flow containing personal information is intercepted, there is
also information disclosure. However, there will leak less information as would
be the case when a data store is involved. Therefore the priority of this kind
of information disclosure is still high priority but lower than the one of the
data store.

\npar Identifiability of stored user data also poses a high risk as it should be
assured that only the customer can access his own identiable information.

\subsection{Medium Priority}

\ %Lay-out

\begin{itemize}
  \item T07 -- Identifiability of bills sent to the Third Party Billing
  Web Service
  \item T02 -- Linking measurementsdata to user data
  \item T01 -- Linking measurementsdata
  \item T06 -- Linkability of bills sent to the Third Party Billing Web
  Service 
  \item T10 -- Non-compliance
  \item T11 -- Missing consumer consents
\end{itemize}

\npar Identifiability for bills has a medium priority since the identification
information included in the bills is rather limited. This identification
information is on the other hand necessary as one needs to be able to know who
has to pay the bill. The risk of identifiability is higher than the one of
linkability since identifiability leads directly to privacy violation. With
linkability is it only a possible side effect.

\npar Linking measurements, user or other data only poses a real threat when the
linking actually leads to identification. Therefore, linkability on its own is
only considerd medium risk. Notice that there are several linking threats. They
are organized according to the size of the involved IOI's. When one is able to
link for example the measurements data to the user data, he/she has much more
information at his/her disposal than linking only bills.

\npar Non-compliance of the system in general, and missing consents specically,
will result in a violation of the consumer's privacy. However, the management
are considered knowledgeable and at least aware of the consequences of ignoring
legislation. Also, even though officially a system is non-compliant, it is still
possible that the general legislation concepts are present. Therefore these
threats have the lowest medium priority.

\subsection{Low Priority}

\ %Lay-out 

\begin{itemize}
  \item T09 -- Content inaccuracy
  \item T08 -- Information disclosure internal process
\end{itemize}

\npar Information inaccuracy is considered low risk. Although inaccurate data
can indeed lead to potential problems (e.g. a consumer who is unaware of the
closure of a valve), this does not form an issue as no important decisions are
made based on this data.

\npar The internal process is considered low priority as there is a trust
relation with the employees. Most likely there is also a non-disclosure
agreement in their contract with associated consequences (lawsuit, enprisonment,
fine, fired, etc.) depending on the severity of the non-disclosure. So given the
trust relationship between the employees and the company, it is less likely that
they will violate the rules. T08 is placed lower in the ordering because it is
less likely to occur.
