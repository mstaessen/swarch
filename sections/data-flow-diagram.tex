\section{Data Flow Diagram}
\label{sec:data-flow-diagram}

\subsection{Level 0}

\begin{figure}[h!]
	\begin{centering}
		\includegraphics[width=0.8\textwidth]{figs/level-0.pdf}
		\caption{The level zero data flow diagram of the ReMeS system.}
		\label{fig:data-flow-diagram-lvl0}
	\end{centering}
\end{figure}

\npar In figure \ref{fig:data-flow-diagram-lvl0} the level zero data flow
diagram (DFD) is depicted. It is a representation of the whole system and is quite analogous
to the context diagram. There is only one process, ReMeS, and all actors are
derived from the context diagram. Notice that ReMeS operator and ReMeS
technician are generalised into ReMeS personnel. This is done because there is
no significant difference between the two from the system's point of view. They
both only communicate with the Operator Portal.

\subsection{Level 1}

\begin{figure}[h!]
	\begin{centering}
		\includegraphics[width=\textwidth]{figs/level-1.pdf}
		\caption{The most detailed data flow diagram of the ReMeS system, namely the
		second level.}
		\label{fig:data-flow-diagram-lvl2}
	\end{centering}
\end{figure}

\npar In the next level, the single ReMeS process is decomposed. The
decomposition follows from the component diagram. 

\npar Notice that the ``Data Processor'' and ``Anomaly Detector'' are not
present in the diagram. Because the Data Processor hands over all trames 
to the Anomaly Detector and the latter's service is only used by the Data
Processor, they are merged into a single ``Process Data'' process.

\npar The database component from the component diagram is also decomposed into
different data stores: Measurements and Information Data, User Data and Alarm
Configuration Data.

\npar The result is shown in figure \ref{fig:data-flow-diagram-lvl2}.



