\section{Data Flow Diagram}
\label{sec:data-flow-diagram}

\subsection{Level 0}

\begin{figure}[H]
	\begin{centering}
		%TODO Figure
		%\includegraphics[width=1.2\textwidth,angle=90]{figs/data-flow-diagram-lvl0.pdf}
		\caption{The level zero data flow diagram of the ReMeS system.}
		\label{fig:data-flow-diagram-lvl0}
	\end{centering}
\end{figure}

\npar In \ref{fig:data-flow-diagram-lvl0} the level zero data flow diagram (DFD)
is depicted. It is representation of the whole system and is quite analogous to
the context diagram. There is only one process, ReMeS, and all actors are
derived from the context diagram. Notice that ReMeS operator and ReMeS
technician are generalised into ReMeS personnel. This is done because there is
no significant difference between the two from the system's point of view. They
both only communicate with the Operator Portal.

\subsection{Level 1}

\begin{figure}[H]
	\begin{centering}
		%TODO Figure
		%\includegraphics[width=1.2\textwidth,angle=90]{figs/data-flow-diagram-lvl1.pdf}
		\caption{The level 1 data flow diagram of the ReMeS system, namely the
		second level.}
		\label{fig:data-flow-diagram-lvl1}
	\end{centering}
\end{figure} 

\npar In the next level of decomposition the components with external interfaces
are introduced. This means nothing else than the components who interect with
components or services from outside the ReMeS system. These components also
follow directly from the context diagram. In addition to these components all
the databases of the architecture are modelled as (separate) data stores. They
are added in this level because they contribute in great matter to the daily
operation of the ReMeS sytem. 

\npar Notice that the ``Data Processor" and ``Anomaly Detector" are not present
in the diagram. They are replaced by the ``Process Data" process. The reason for
doing this is simply because the Data Processor hands all trames over to the
Anomaly Detector and the latter's service is only used by the Data Processor.
Therefore, with no loss of generalisation they can modelled as a single process
since no threats are lost. Notice that they both require interaction with the
database but this can be left out of consideration for the analysis of threats. 
A schematic overview is given in figure \ref{fig:data-flow-diagram-lvl1}.

\subsection{Level 2}

\begin{figure}[H]
	\begin{centering}
		%TODO Figure
		%\includegraphics[width=1.2\textwidth,angle=90]{figs/data-flow-diagram-lvl2.pdf}
		\caption{The most detailed data flow diagram of the ReMeS system, namely the
		second level.}
		\label{fig:data-flow-diagram-lvl2}
	\end{centering}
\end{figure}

\npar Level two represents the final decomposition. In this last step all
other internal components are added to the diagram. The result is shown in
figure \ref{fig:data-flow-diagram-lvl2}.
