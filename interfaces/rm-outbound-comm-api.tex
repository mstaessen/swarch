\section{OutboundCommAPI}
\label{api:rm-outbound-comm-api}

\subsection{Interface Identity}

\npar \interface{OutboundCommAPI} provides methods for notifying the
implementor.

\subsection{\method{send(OCCommand)}}

\subsubsection{Resources Provided}

\begin{quote}
	\begin{description}
		\item[Syntax] \
		\begin{verbatim}
void send(OCCommand command)
    throws NullPointerException, IllegalOCCommandException
		\end{verbatim}
		\item[Arguments] \
		\begin{itemize}
		  \item command: contains all the data required to send a trame towards a
		  remote module.
		\end{itemize}
		\item[Effect] Invocation of this method has a twofolded result. First of
		all is the given command translated into a (physical) trame and second the
		trame is sent towards the remote module (using the information in the
		command).
		\item[Restrictions] \
		\begin{itemize}
		  \item The given command can not be the null reference.
		  \item The given command has to contain all the data to allows the
		  implementator of this interface to construct the trame and sent it towards
		  the remote module. This means that it has to at least contain a recipient
		  (i.e. remote module id) and an actual message.
		\end{itemize}
	\end{description} 
\end{quote}

\subsubsection{Data Types Definition}

\begin{quote}
	\begin{description}
		\item[OCCommand] This object wraps a request to send a control
		trame to a given remote device.
	\end{description} 
\end{quote}

\subsubsection{Exception Definition} 

\begin{quote}
	\begin{description}
		\item[NullPointerException] This exception is thrown when the given command is
		the null reference.
		\item[IllegalCommandException] This exception is thrown when there is not
		enough information in the command to sent the trame. For example the lacking
		of a recipient or actual data.
	\end{description} 
\end{quote}