\section{TranslatorAPI}
\label{api:rm-translator-api}

\subsection{Interface Identity}

\npar The \interface{TranslatorAPI} offers functionality for the translation of
incoming and outgoing trames.

\subsection{\method{translateFromDevice(RawTrame)}}

\subsubsection{Resources Provided}

\begin{quote}
	\begin{description}
		\item[Syntax] \
		\begin{verbatim}
translateFromDevice(byte[] rawTrame) : Message
    throws IllegalFormatException
		\end{verbatim}
		\item[Arguments] \
		\begin{itemize}
		  \item the rawTrame is the trame to be translated into a message for further
		  usage in ReMeS.
		\end{itemize}
		\item[Effect] The given parameter will be translated into a Message object.
		This means that all information, relevant for further processing, will be
		extracted of the parameter and placed into a new object.
		\item[Restrictions] \
		\begin{itemize}
		  \item The rawTrame should be in the right format.
		\end{itemize}
	\end{description} 
\end{quote}

\subsubsection{Data Types Definition}

\begin{quote}
	\begin{description}
		\item[Message] This object represents an objectified trame. It contains
		information about the customer (i.e. the ``owner'' of the trame) and the data
		of the trame.
	\end{description} 
\end{quote}

\subsubsection{Exception Definition} 

\begin{quote}
	\begin{description}
		\item[IllegalFormatException] This exception will be thrown when the bytes of
		the incoming trame are not ordered (or filled in) as they should be.
	\end{description} 
\end{quote}

\subsection{\method{translateToDevice(Message)}}

\subsubsection{Resources Provided}

\begin{quote}
	\begin{description}
		\item[Syntax] \
		\begin{verbatim}
translateToDevice(Message message) : byte[]
    throws IllegalContentException, NullPointerException
		\end{verbatim}
		\item[Arguments] \
		\begin{itemize}
		  \item message, this is the message that will be translated.
		\end{itemize}
		\item[Effect] The given parameter will be translated into an array of bytes.
		This means that all information, relevant for a trame, will be extracted of
		the parameter and placed into the array according to a predefined format.
		\item[Restrictions] \
		\begin{itemize}
		  \item All the fields required for constructing the array should be present
		  in the message.
		  \item The given object can't be the null reference.
		\end{itemize}
	\end{description} 
\end{quote}

\subsubsection{Data Types Definition}

\begin{quote}
	\begin{description}
		\item[Message] This object represents an objectified trame. It contains
		information about the customer (i.e. the ``owner'' of the trame) and the data
		of the trame.
	\end{description} 
\end{quote}

\subsubsection{Exception Definition} 

\begin{quote}
	\begin{description}
		\item[IllegalContentException] This exception will be thrown when one or more
		fields of a given message parameter are not filled in. 
		\item[NullPointerException] This exception will be thrown when the null
		reference is used as a parameter to this method call.
	\end{description} 
\end{quote}