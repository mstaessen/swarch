\section{AnomalyDetectionAPI}
\label{api:anomaly-detection-api}

\subsection{Interface Identity}

\npar \interface{AnomalyDetectionAPI} provides methods to dispatch anomaly
detection commands to the intended instances.

\subsection{\method{execute(ADCommand)}}

\subsubsection{Resources Provided}

\begin{quote}
	\begin{description}
		\item[Syntax] \ 
		\begin{verbatim}
void execute(ADCommand command) 
    throws NullPointerException;
		\end{verbatim}
		\item[Arguments] \
		\begin{itemize}
			\item command: represents a request to run the anomaly detection algorithms
			for a given trame that is encapsulated in the command.
		\end{itemize}
		\item[Effect] An anomaly detection instance is removed from the ADInstancePool
		and the command is executed on that instance. The results of the anomaly will
		be returned asynchronously. 
		\item[Restrictions] \ 
		\begin{itemize}
			\item The given command is not null.
			\item This method has to be implemented asynchronously, i.e. the method will
			not be blocking.
		\end{itemize}
	\end{description} 
\end{quote}

\subsubsection{Data Types Definition}

\begin{quote}
	\begin{description}
		\item[ADCommand] This object wraps a request to perform anomaly detection for
		the device that sent the trame.	
	\end{description} 
\end{quote}

\subsubsection{Exception Definition}

\begin{quote}
	\begin{description}
		\item[NullPointerException] This exception is thrown if the command is null.
	\end{description} 
\end{quote}