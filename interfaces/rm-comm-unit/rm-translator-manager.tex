\section{TranslatorManager}
\label{api:rm-translator-manager}

\subsection{Interface Identity}

\npar The \interface{TranslatorManager} provides methods for retrieving
Translators.

\subsection{\method{GetTranslatorFor(RemoteDevice)}}

\subsubsection{Resources Provided}

\begin{quote}
	\begin{description}
		\item[Syntax] \
		\begin{verbatim}
Translator GetTranslatorFor(RemoteDevice device) 
    throws UnknownRemoteDeviceException
		\end{verbatim}
		\item[Arguments] \
		\begin{itemize}
		  \item device, based on this parameter the correct Translator will be
		  fetched.
		\end{itemize}
		\item[Effect] Upon calling this method the parameter is analyzed and based on
		the trame format the correct translator is returned.
		\item[Restrictions] \
		\begin{itemize}
		  \item The device (more specific the trame format) should be
			known.
		\end{itemize}
	\end{description} 
\end{quote}

\subsubsection{Data Types Definition}

\begin{quote}
	\begin{description}
		\item[RemoteDevice] This object contains all sorts of information regarding
		remote modules. For example, id number, used communication channel, trame
		format, etc.
		\item[Translator] An instance of this class is a wrapper for the translator
		functionality. It offers a TranslatorAPI interface, see section
		\ref{api:rm-translator-api}.
	\end{description} 
\end{quote}

\subsubsection{Exception Definition} 

\begin{quote}
	\begin{description}
		\item[UnknownRemoteDeviceException] This exception will be thrown whenever a
		remote device with an unknown trame format is handed as a parameter (or when
		its the null reference).
	\end{description} 
\end{quote}