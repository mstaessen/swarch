\section{OCQueueAPI}
\label{api:oc-queue-api}

\subsection{Interface Identity}

\npar \interface{OCQueueAPI} provides methods to enqueue and dequeue
OCCommands in the OCQueue.

\subsection{\method{enqueue(OCCommand, double)}}

\subsubsection{Resources Provided}

\begin{quote}
	\begin{description}
		\item[Syntax] \ 
		\begin{verbatim}
void enqueue(OCCommand command, double priority) 
    throws NullPointerException;
		\end{verbatim}
		\item[Arguments] \
		\begin{itemize}
			\item command: contains an encapsulated request send out a control trame to
			a certain device.
			\item priority: the priority this command will have in queue. 
		\end{itemize}
		\item[Effect] The command is inserted in the queue accoring to its priority.
		\item[Restrictions] \ 
		\begin{itemize}
			\item The given command is not null.
			\item A greater number corresponds to a higher priority. 
		\end{itemize}
	\end{description} 
\end{quote}

\subsubsection{Data Types Definition}

\begin{quote}
	\begin{description}
		\item[OCCommand] This object wraps a request to send a control trame to a
		given remote device.
	\end{description} 
\end{quote}

\subsubsection{Exception Definition}

\begin{quote}
	\begin{description}
		\item[NullPointerException] This exception is thrown if the command is null.
	\end{description} 
\end{quote}

\subsection{\method{dequeue()}}

\subsubsection{Resources Provided}

\begin{quote}
	\begin{description}
		\item[Syntax] \ 
		\begin{verbatim}
OCCommand dequeue() 
    throws EmptyQueueException;
		\end{verbatim}
		\item[Arguments] This method takes no arguments.
		\item[Effect] The OCCommand with the highest priority is popped of the queue.
		\item[Restrictions] \ 
		\begin{itemize}
			\item The queue is not empty. 
		\end{itemize}
	\end{description} 
\end{quote}

\subsubsection{Data Types Definition}

\begin{quote}
	\begin{description}
		\item[OCCommand] This object wraps a request to send a control trame to a
		given remote device.
	\end{description} 
\end{quote}

\subsubsection{Exception Definition}

\begin{quote}
	\begin{description}
		\item[EmptyQueueException] This exception is thrown if the queue is empty.
	\end{description} 
\end{quote}

