\section{AlarmAPI}
\label{api:alarm-api}

\subsection{Interface Identity}

\npar \interface{AlarmAPI} provides methods for the retrieval of and storage of
alarm (information).

\subsection{\method{execute(AlarmQuery)}}

\subsubsection{Resources Provided}

\begin{quote}
	\begin{description}
		\item[Syntax] \
		\begin{verbatim}
AlarmResult execute(AlarmQuery query)
    throws IllegalQuerySyntaxException, NullPointerException;
		\end{verbatim}
		\item[Arguments] \
		\begin{itemize}
		  \item query: the query to be executed on the database. 
		\end{itemize}
		\item[Effect] The query is executed on the database and the result is returned
		as an AlarmResult. 
		\item[Restrictions] \
		\begin{itemize}
		  \item The query can't be the null reference.
		  \item The syntax of the query has to be correct.
		\end{itemize}
	\end{description} 
\end{quote}

\subsubsection{Data Types Definition}

\begin{quote}
	\begin{description}
		\item[AlarmResult] This object is a container for multiple alarms. 
		\item[AlarmQuery] This object is a wrapper for queries only executable on a
		part of the shared repository, namely the alarm storage.
		\item[Alarm] This object is a container for all sorts of information useful
		for sending an alarm. This includes one or more recipients, the problem or reason,
		etc.
	\end{description} 
\end{quote}

\subsubsection{Exception Definition} 

\begin{quote}
	\begin{description}
		\item[NullPointerException] This exception is thrown when the given parameter
		is the null reference.
		\item[IllegalQuerySyntaxException] This exception is thrown when the syntax of
		the given query is invalid.
	\end{description} 
\end{quote}