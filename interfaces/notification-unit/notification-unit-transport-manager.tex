\section{TransportManager}
\label{api:notification-unit-transport-manager}

\subsection{Interface Identity}

\npar The \interface{TransportManager} provides methods for retrieving a
Transport object.

\subsection{\method{getTransport(CommunicationChannel)}}

\subsubsection{Resources Provided}

\begin{quote}
	\begin{description}
		\item[Syntax] \
		\begin{verbatim}
Transport getTransport(CommunicationChannel channel)
    throws NullPointerException
		\end{verbatim}
		\item[Arguments] \
		\begin{itemize}
		  \item channel, the content of this object will be used to get right
		  Transport.
		\end{itemize}
		\item[Effect] Based on the given channel the right Transport object will be
		fetched and returned. 
		\item[Restrictions] \
		\begin{itemize}
		  \item The given channel can not be the null reference. 
		\end{itemize}
	\end{description} 
\end{quote}

\subsubsection{Data Types Definition}

\begin{quote}
	\begin{description}
		\item[CommunicationChannel] This object is a container for an enumeration of
		the different physical communication channels (for example WiFi, etc.)
		\item[Transport] This object is responsible for the sending of a given
		message.
	\end{description} 
\end{quote}

\subsubsection{Exception Definition} 

\begin{quote}
	\begin{description}
		\item[NullPointerException] This exception will be thrown when the given
		argument is null.
	\end{description} 
\end{quote}