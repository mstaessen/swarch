\section{TransportAPI}
\label{api:notification-unit-transport-api}

\subsection{Interface Identity}

\npar \interface{TransportAPI} provides methodes for the (physical) sending of
an alarm.

\subsection{\method{send(Alarm)}}

\subsubsection{Resources Provided}

\begin{quote}
	\begin{description}
		\item[Syntax] \
		\begin{verbatim}
send(Alarm alarm) : void
    throws NullPointerException, IllegalAlarmException
		\end{verbatim}
		\item[Arguments] \
		\begin{itemize}
		  \item the contents of the alarm will be used to send the alarm (physically). 
		\end{itemize}
		\item[Effect] The contents of the alarm message are sent to the recipients
		which are also contained in the alarm message.
		\item[Restrictions] \
		\begin{itemize}
		  \item The given parameter can not be the null reference.
		  \item There has to be at least one recipient in the Alarm.
		  \item The message part of the Alarm can't be empty.
		\end{itemize}
	\end{description} 
\end{quote}

\subsubsection{Data Types Definition}

\begin{quote}
	\begin{description}
		\item[Alarm] This class is a container for all sorts of information useful for
		sending an alarm. This includes one or more recipients, the problem or reason,
		etc.
	\end{description} 
\end{quote}

\subsubsection{Exception Definition} 

\begin{quote}
	\begin{description}
		\item[NullPointerException] This exception will be thrown when the given Alarm
		is null.
		\item[IllegalAlarmException] This exception will be thrown when there is no
		recipient or reason incuded in the Alarm.
	\end{description} 
\end{quote}