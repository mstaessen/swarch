\section{TemporaryBufferAPI}
\label{api:temporary-buffer-api}

\subsection{Interface Identity}

\npar \interface{TemporaryBufferAPI} provides methods to temporarily store 
QueryCommands (first in, first out) when the measurements database is
unavailable.

\subsection{\method{add(QueryCommand)}}

\subsubsection{Resources Provided}

\begin{quote}
	\begin{description}
		\item[Syntax] \ 
		\begin{verbatim}
void add(QueryCommand command) 
    throws NullPointerException;
		\end{verbatim}
		\item[Arguments] \
		\begin{itemize}
			\item command: contains an encapsulated write query to be executed on the
			measurements database.
		\end{itemize}
		\item[Effect] The command is added to the buffer. 
		\item[Restrictions] \ 
		\begin{itemize}
			\item Only QueryCommands that represent write operations are accepted (read
			queries can use the cache). 
		\end{itemize}
	\end{description} 
\end{quote}

\subsubsection{Data Types Definition}

\begin{quote}
	\begin{description}
		\item[QueryCommand] This object wraps a query to the measurements database. It
		provides a Callback to return the results to the command issuer.
	\end{description} 
\end{quote}

\subsubsection{Exception Definition}

\begin{quote}
	\begin{description}
		\item[NullPointerException] This exception is thrown if the command is null.
	\end{description} 
\end{quote}

\subsection{\method{remove()}}

\subsubsection{Resources Provided}

\begin{quote}
	\begin{description}
		\item[Syntax] \ 
		\begin{verbatim}
QueryCommand remove() 
    throws EmptyBufferException;
		\end{verbatim}
		\item[Arguments] This method takes no arguments.
		\item[Effect] The QueryCommand at the front is popped of the buffer.
		\item[Restrictions] \ 
		\begin{itemize}
			\item The buffer is not empty. 
		\end{itemize}
	\end{description} 
\end{quote}

\subsubsection{Data Types Definition}

\begin{quote}
	\begin{description}
		\item[QueryCommand] This object wraps a query to the measurements database. It
		provides a Callback to return the results to the command issuer.
	\end{description} 
\end{quote}

\subsubsection{Exception Definition}

\begin{quote}
	\begin{description}
		\item[EmptyBufferException] This exception is thrown if the buffer is empty.
	\end{description} 
\end{quote}

