\method{markPaid(InvoiceId : String) : void}

\section{BillingCommunicator}

\subsection{Interface Identity}

\subsection{\method{markPaid(InvoiceId)}}

\subsubsection{Resources Provided}

\begin{quote}
	\begin{description}
		\item[Syntax] \
		\begin{verbatim}
void markPaid(String InvoiceId)
    throws NullPointerException, IllegalInvoiceIdException
		\end{verbatim}
		\item[Arguments] \
		\begin{itemize}
		  \item id, this unique id identifies an invoice throughout the system.
		\end{itemize}
		\item[Effect] The invoice with the given id is marked as paid and this
		information will also be stored. 
		\item[Restrictions] \
		\begin{itemize}
		  \item The given id can't be the null reference.
		  \item The given id has to refer to an existing invoice.
		  \item This method can only be called when the payment for the invoice with
		  the given id is received.
		\end{itemize}
	\end{description} 
\end{quote}

\subsubsection{Data Types Definition}

\begin{quote}
	\begin{description}
		\item[]There are no new data types introduced in this interface. 
	\end{description} 
\end{quote}

\subsubsection{Exception Definition} 

\begin{quote}
	\begin{description}
		\item[NullPointerException] 
		\item[IllegalInvoiceIdException] This exception is thrown when 
	\end{description} 
\end{quote}