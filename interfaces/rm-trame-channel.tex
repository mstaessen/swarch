\section{TrameChannel}
\label{api:rm-trame-channel}

\subsection{Interface Identity}

\npar The \interface{TrameChannel} provides methods for the communication of
events as the \emph{Publisher-Subscriber} pattern prescribes.

\subsection{\method{publish(CommunicationEvent)}}

\subsubsection{Resources Provided}

\begin{quote}
	\begin{description}
		\item[Syntax] \
		\begin{verbatim}
publish(CommunicationEvent communicationEvent) : void
    throws NullPointerException
		\end{verbatim}
		\item[Arguments] \
		\begin{itemize}
		  \item communicationEvent, this event will be published towards subscribers.
		\end{itemize}
		\item[Effect] Subscribers to receiving CommunicationEvents will be notified of
		the availability of a new event. 
		\item[Restrictions] \
		\begin{itemize}
		  \item The given parameter can not be the null reference.
		\end{itemize}
	\end{description} 
\end{quote}

\subsubsection{Data Types Definition}

\begin{quote}
	\begin{description}
		\item[CommunicationEvent] This object is a wrapper for a Message object (see
		section \ref{api:rm-translator-api}).
	\end{description} 
\end{quote}

\subsubsection{Exception Definition} 

\begin{quote}
	\begin{description}
		\item[NullPointerException] This exception will be thrown when the given
		CommunicationEvent is the null reference.
	\end{description} 
\end{quote}

\subsection{\method{subscribe(Filter)}}

\subsubsection{Resources Provided}

\begin{quote}
	\begin{description}
		\item[Syntax] \
		\begin{verbatim}
subscribe(Filter filter) : void
    throws NullPointerException
		\end{verbatim}
		\item[Arguments] \
		\begin{itemize}
		  \item The content of the given filter will be used for filtering
		  notifications of arrivals of events.
		\end{itemize}
		\item[Effect] The subscribing object will be marked as a subscriber for
		certain events (this is determined on the filter).
		\item[Restrictions] \
		\begin{itemize}
		  \item The given parameter can not be the null reference.
		\end{itemize}
	\end{description} 
\end{quote}

\subsubsection{Data Types Definition}

\begin{quote}
	\begin{description}
		\item[Filter] This object offers functionality to indicate what type of events
		one wants to receive.
	\end{description} 
\end{quote}

\subsubsection{Exception Definition} 

\begin{quote}
	\begin{description}
		\item[NullPointerException] This exception will be thrown when the given
		Filter is the null reference.
	\end{description} 
\end{quote}
