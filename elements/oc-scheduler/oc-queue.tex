\section{OCQueue}
\label{element:oc-queue}

\subsection{Roles and responsibilities}

\npar The OCQueue does nothing else than provide a priority queue for
OCCommands with accompanying insert and retrieve functionality. Since
it is a priority queue the commands can be placed in different places in the
queue depending on their priority. More precisely, for each priority level will
the index of the last element with that priority be kept in a list. When a new
command needs to be inserted, the priority of that command is determined.
Subsequently, the priority is used to retrieve the index of the last element and
in this way the new command can be inserted.

\subsection{Provided interfaces}

\begin{itemize}
  \item OCQueueAPI, see \ref{api:oc-queue-api}.
\end{itemize}