\chapter{Understanding the architecture}
\label{chap:understanding-architecture}

\npar All incoming communication goes through the Incoming Communication
Component. The received bits are converted into an alarm trame and are
subsequently handed over to the right component. In this case, it is the Alarm
Processor. This can be done through the Receive Alarm Trame interface. The Alarm
Processor will determine whether the remote module that sent the alarm has a
control valve associated with it (by retrieving the configuration data from the
Alarm Configuration Database). If so, the Actuator Controller is invoked through
its ActivateValve interface. Afterwards the alarm will be stored in the
measurements database (by constructing a query and invoking the storeData(Query)
method).

\npar The Actuator Controller will construct a command to shut down the
corresponding valve and invoke the Outgoing Communication Component’s
sendCommand() method. It is important that this command arrives at the remote
module so an acknowledgement is be sent by the module upon receiving the
command. When the acknowledgement does not arrive within a certain interval, the
command is resent. If this happens several times in a row, a notification is
sent towards the customer to inform him of the multiple failures. The activation
of the valve is stored in the measurement and information database. All these
responsibilties belong to the Actuator Controller.

\npar Finally, an alarm notification is sent towards the customer (and/or
emergency services) by using the Outgoing Communication Component. In the
Outgoing Communication Component the different messages are sent according to
their priority. The command received from the Actuator Controller in this
case has a high priority. To notify all the stakeholders, the recipient
information is fetched from the database. The notification is also persisted.
