\chapter{Additional constraints}
\label{additional-constraints}

\begin{itemize}
	\item A utility providing company can only manage modules for the utilities it
	provides.
	\item The customer of every event is equal to the customer that belongs to the
	remote module.
	\item Modules that depend on each other (monitor/valve combinations, multiple
	monitoring modules on the same line, \ldots) should always concern the same
	utility.
	\item WiFiModules should always use an external power because the WiFi.
	connection consumes to much electricity.
	\item Only battery powered modules can send low battery alarms.
	\item The update frequency on battery powered devices can only be once per 24 hours. 
	\item The cumulative values of subsequent measurements of a remote
	monitoring module should rise monotonically except when the monitoring module
	is installed on a utility that can supply to the grid (e.g. houses with solar
	energy).
	\item No access is granted to the ReMeS portal without authentication.
	\item No access to personal data in the ReMeS portal is allowed without
	authorization.
	\item No valves should be shut without a previous anomaly detection.
	\item No alarms should be triggered without a anomaly detection.
	\item In case of a gas leak, the customer and the emergency services should be
	notified.
	\item In case of a water leak, only the customer should be notified.
	\item A customer cannot receive invoices from a utility providing company with
	whom the customer has no contract.
	\item In order to create a user profile for ReMeS, the utility company should
	participate in the ReMeS program.
\end{itemize}