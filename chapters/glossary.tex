\chapter{Glossary}
\label{chap:glossary}

\begin{description}
\item[Advice] An \emph{advice} is information provided to the customer in order
to reduce one's \emph{utility} bill. This information is typically obtained through
analysis of one's consumption.

\item[Alarm] An \emph{Alarm} is a message sent by ReMeS to inform the customer
that some anomaly is detected. This could be e.g. an excessive usage of a cetrain
\emph{utility}.

\item[AlarmCall] An \emph{AlarmCall} represents the telephone conversation
between ReMeS and the \emph{Customer}.

\item[AlarmConfiguration] An \emph{AlarmConfiguration} is a general name for the
contact details of a \emph{customer}. \emph{Customers} can be reached via sms or
email.

\item[AlarmEvent] An \emph{AlarmEvent} is an event which is sent from the
\emph{CommunicationUnit} to the \emph{DataCenter} when the latter receives an
\emph{AlarmTrame}. This is a single format that every unit of the
\emph{DataCenter} can read.

\item[AlarmTrame] This is a \emph{Trame} send by a \emph{module} to report an
\emph{alarm}. This could be in any format, depending on the \emph{module}.

\item[AnalysticsUnit] The \emph{AnalyticsUnit} is responsible for the analysis
of all the monitoring data originating from the \emph{customers}. This analysis
includes several algorithms to generate detailed consumption patterns,
\emph{advices} and \emph{reports}. 

\item[AnomalyDectectionUnit] The \emph{AnomalyDectectionUnit} runs several
Machine Learning algorithms to detect leaks and other anomalies. Furthermore any
\emph{alarms} generated by \emph{modules} are passed to this unit so it can take
appropriate steps (i.e. generate an \emph{alarm}).

\item[Authorized User] An \emph{Authorized User} is a user that can make changes
in the ReMeS system on behalf of other customers (e.g. a Technician) or on
behalf of itself (e.g. the customer) %TODO

\item[BoughtModule] The \emph{boughtModule} refers to \emph{modules} which are
bought by utility companies when they decide to not outsource the billing aspect
to ReMeS.

\item[BusinessCustomer] A \emph{BusinessCustomer} is a company which is a
\emph{customer} (of ReMes and/or several utility companies). The company is a
consumer of \emph{utilities} and so not necessarily a utility company itself.

\item[CallCenter] This is the contact center for troubleshooting and notifying
customers. \emph{Alarms} which are 

\item[Company] A \emph{Company} is in the context of this project a
\emph{utility} company. A \emph{Company} can provide more than \emph{utility}. 

\item[CommunicationUnit] This unit is responsible for the collection of data
from all the \emph{modules}. Data could be in different formats (dependent on
the \emph{module}) and could be received through various communication media.

\item[Contract] A Contract is negotiated between a \emph{Customer} and a
\emph{Company} and connects both parties inextricable for a certain period of
time. When one of the two parties wishes to terminate this \emph{Contract}
premature, they typically will have to have a valid reason also included in the
\emph{Contract}. Furthermore there is typically information included about the
price per unit per utility.

\item[Configuration] A \emph{Configuration} denotes the tuning of a
\emph{module}. This could e.g. the automatic closure of valves in case of an
anomaly.

\item[Customer] A \emph{Customer} is a recipient of a \emph{utility} in exchange
for a payment based on the his/her or it's consumption. There are two kinds of
\emph{Customers}: \emph{ResidentialCustomers} \emph{BusinessCustomers}.

\item[DataCenter] \emph{DataCenter} is a general name for all the different
units in the system. This is a clever notation to easily refer to AnalyticsUnit,
PredictionUnit, StorageUnit and AnomalyDetectionUnit in one time.

\item[Electricity] \emph{Electricity} is used in this context as the public
service, \emph{Electricity}.

\item[EmergencyCall] An \emph{EmergencyCall} is special kind of
\emph{AlarmCall}, namely to the \emph{EmergencyService}.

\item[EmergencyService] Emergency services are organizations which ensure public
safety and health by addressing different emergencies. The three main services
are: Police, Fire Department and Medical urgency.

\item[Event] An event is a message which is send from the
\emph{CommunicationUnit} to one of the units of the \emph{DataCenter} (it is a
one-way traffic message, so it is not send in the other direction). It forms a
generalisation for \emph{UpdateEvents} and \emph{AlarmEvents}.

\item[Gas] \emph{Gas} is used in this context as the public service,
\emph{Gas}.

\item[Government] The \emph{Government} is a stakeholder of ReMeS with as main
concern having insight in the usage statistics of utilities in the country.
Foreign \emph{governmnent} are also stakeholders if \emph{utilities} are
exported or imported.

\item[GPRSModule] This refers to a specific kind of \emph{Module} which
communicates with the \emph{CommunicationUnit} through the use of GPRS
technology.

\item[HiredModule] A \emph{HiredModule} is a \emph{Module} hired by a
\emph{customer}. This occurs when a \emph{company} has outsourced the billing
aspect to ReMeS.

\item[Invoice] An Invoice is a document containing the consumption of a
\emph{Customer} for a certain utility during a certain period of time.
Furthermore (and more important), this document also contains the total amount
the \emph{Customer} has to pay for this consumption to the \emph{Company}.

\item[IPCommunicationUnit] This unit is responsible for the collection of
measurement data through GPRS and Wifi. The collected data is then send to the
DataCenter.

\item[Module] This is a general name for all sorts of modules.
There are \emph{Modules} who measure monitor data and \emph{Modules} that can
control the flow (valves). Another way in which \emph{Modules} can differ is the
communication type (i.e. GPRS, SMS and Wifi). A final distinction is the owner,
modules can be owned by a \emph{Company} or hired by a customer.

\item[MonitoringModule] This is a \emph{Module} which measures the consumption
of a certain \emph{utility}. The measured data is send to the
\emph{CommunicationUnit} in a certain format, dependent of the type of the
module.

\item[Operator] An Operator is a person who works in the \emph{CallCenter}. The
task of an \emph{Operator} is dual. First of all to assist \emph{Customers} in
troubleshooting. Secondly, any kind of notification for a \emph{Customers} is
done by such a person.

\item[Payment] A payment is the transfer of money, using a certain payment
method (cash, creditcard, etc.), from one party (such as a person or company) to
another.

\item[Personnel] This is a general name for all different kind of Employees. In
this project there are \emph{SystemAdministrators}, \emph{Operators} and
\emph{Technicians}.

\item[Prediction] \emph{Companies} want to know how much of which utility they
will have to produce in the coming period (this could a week, a month, a year,
etc.). To Provide \emph{Companies} with this information accurate
\emph{Predictions} are made which contain exactly the previously mentioned
information.

\item[PredictionUnit] This unit is responsible for generating accurate
\emph{Predictions} based on the measurement data originating from the clients.
This data comes from a \emph{CommunicationUnit}.

\item[ReMeS Portal] % TODO

\item[Report] \emph{Stakeholders} of ReMeS are interested in its activities and
consumption patterns of its users. To provide the stakeholders with the
appropriate information, \emph{reports} are published on fixed data.

\item[ResidentialCustomer] A \emph{ResidentialCustomer} is a single person which
is a \emph{Customer} of a \emph{utility} \emph{Company}. 

\item[SIMCard] A SIM Card (subscriber identification module) contains the
account information for mobile numbers such as telephone number, serial number
and default carrier, etc.

\item[SIMCardModule] A \emph{SIMCardModule} is a \emph{Module} which
communicates with the \emph{CommunicationUnit} using a phonenumber. This is
realized via a \emph{SIMCard} placed in the \emph{Module}. The
\emph{SIMCardModule} is in fact a generalisation for \emph{SMSModules} and
\emph{GPRSModules}.

\item[SMSCommunicationUnit] This unit is responsible for the collection of
measurement data through SMS. The collected data is then send to the DataCenter.

\item[SMSModule] This refers to a specific kind of \emph{Module} which
communicates with the \emph{CommunicationUnit} through the use of SMS.

\item[StorageUnit] The \emph{StorageUnit} is responsible for the persistent
storage of all \emph{events}. The other units can collect their data in these
persistent storage.

\item[StakeHolder]A \emph{Stakeholder} is a party that can affect or be
affected by the actions a company, in this case ReMeS.

\item[SystemAdministrator] A \emph{SystemAdministrator} is a manager of a
\emph{CommunicationUnit} and is responsible for uptime and smooth operations.

\item[Technician] A \emph{Technician} is responsible for the (un)installation
and reparation of Modules and other hardware regarding the various units of the
\emph{DataCenter}. Furthermore he has more priviliges regarding the ReMeS portal
to modify settings of customers.

\item[TelephoneCompany] A \emph{TelephoneCompany} is a provider of certain
communication services, such as (wireless) telephony.

\item[ThirdPartyPaymentService] This is a service for the electronic management
of invoicing. It is in fact an intermediate between client and customer to
collect the payments of any pending invoices.

\item[Trame] A \emph{Trame} is a block of bits that contains certain meta-data
and a data section in a predefined format. This format is different for each
\emph{Module}. The data section can be used in two different ways. One is for
communication from the \emph{Module} to ReMeS, in this case the data section
contains actual measurements. The other way is for communication in the other
direction, then it contains commands.

\item[UpdateEvent] An \emph{UpdatemEvent} is an event which is sent from the
\emph{CommunicationUnit} to the \emph{DataCenter} when the latter receives a
normal data \emph{Trame}. This is a single format that every unit of the
\emph{DataCenter} can read.

\item[UsageDetail] The \emph{UsageDetail} contains a detailed description of the
consumption of a certain \emph{Customer}. This information is particularly
useful for the invoicing.

\item[Utility] is a generalisation of \emph{Water}, \emph{Gas},
\emph{Electricity}.

\item[ValveOperator] A \emph{ValveOperator} is a person or machine who is
granted to shut a valve in case of emergencies or anomalies.

\item[ValveModule] This is a \emph{Module} which controls the flow of a certain
\emph{utility}. The behaviour of this \emph{Module} can be controlled by sending
commands to configure it. An example is the automic sealing when a leak is
detected.

\item[Water] \emph{Water} is used in this context as the public service,
\emph{Water}.

\item[WiFiModule] A \emph{WiFiModule} refers to a specific kind of \emph{Module}
which communicates with the \emph{CommunicationUnit} through the use of WiFi.

\end{description}
