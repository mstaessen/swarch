\chapter{Glossary}
\label{glossary}

\begin{description}
\item[Advice] An \emph{advice} is information provided to the customer in order
to reduce one's \emph{utility} bill. This information is typically obtained through
analysis of one's consumption.

\item[Alarm] An \emph{Alarm} is a message sent by ReMeS to inform the customer
that some anomaly is detected. This could be e.g. an excessive usage of a cetrain
\emph{utility}.

\item[AlarmCall]

\item[AlarmConfiguration] An \emph{AlarmConfiguration} is a general name for the
contact details of a \emph{customer}. \emph{Customers} can be reached via sms or
email.

\item[AlarmEvent] An \emph{AlarmEvent} is an event which is sent from the
\emph{CommunicationUnit} to the \emph{DataCenter} when the latter receives an
\emph{AlarmTrame}. This is a single format that every unit of the
\emph{DataCenter} can read.

\item[AlarmTrame] This is a \emph{Trame} send by a \emph{module} to report an
\emph{alarm}. This could be in any format, depending on the \emph{module}.

\item[AnalysticsUnit] The \emph{AnalyticsUnit} is responsible for the analysis
of all the monitoring data originating from the \emph{customers}. This analysis
includes several algorithms to generate detailed consumption patterns,
\emph{advices} and \emph{reports}. 

\item[AnomalyDectectionUnit] The \emph{AnomalyDectectionUnit} runs several
Machine Learning algorithms to detect leaks and other anomalies. Furthermore any
\emph{alarms} generated by \emph{modules} are passed to this unit so it can take
appropriate steps (i.e. generate an \emph{alarm}).

\item[BoughtModule] The \emph{boughtModule} refers to \emph{modules} which are
bought by utility companies when they decide to not outsource the billing aspect
to ReMeS.

\item[BusinessCustomer] A \emph{BusinessCustomer} is a company which is a
\emph{customer} (of ReMes and/or several utility companies). The company is a
consumer of \emph{utilities} and so not necessarily a utility company itself.

\item[CallCenter] This is the contact center for troubleshooting and notifying
customers. \emph{Alarms} which are 

\item[Company] A \emph{Company} is in the context of this project a
\emph{utility} company. A \emph{Company} can provide more than \emph{utility}. 

\item[CommunicationUnit] This unit is responsible for the collection of data
from all the \emph{modules}. Data could be in different formats (dependent on
the \emph{module}) and could be received through various communication media.

\item[Contract] A Contract is negotiated between a \emph{Customer} and a
\emph{Company} and connects both parties inextricable for a certain period of
time. When one of the two parties wishes to terminate this \emph{Contract}
premature, they typically will have to have a valid reason also included in the
\emph{Contract}. Furthermore there is typically information included about the
price per unit per utility.

\item[Configuration] A \emph{Configuration} denotes the tuning of a
\emph{module}. This could e.g. the automatic closure of valves in case of an
anomaly.

\item[Customer] A \emph{Customer} is a recipient of a \emph{utility} in exchange
for a payment based on the his/her or it's consumption. There are two kinds of
\emph{Customers}: \emph{ResidentialCustomers} \emph{BusinessCustomers}.

\item[DataCenter] \emph{DataCenter} is a general name for all the different
units in the system. This is a clever notation to easily refer to AnalyticsUnit,
PredictionUnit, StorageUnit and AnomalyDetectionUnit in one time.

\item[Electricity] This is the \emph{utility}, electricity.

\item[EmergencyCall] An \emph{EmergencyCall} is special kind of
\emph{AlarmCall}, namely to the \emph{EmergencyService}.

\item[EmergencyService]
\item[Event]
\item[Gas]
\item[Government]
\item[GPRSModule]
\item[HiredModule]
\item[IPCenter]
\item[Invoice]
\item[Module]
\item[MonitoringModule]
\item[Operator]
\item[Payment]
\item[Personnel]
\item[Prediction]
\item[PredictionUnit]
\item[Report]
\item[ResidentialCustomer]
\item[SIMCard]
\item[SIMCardModule]
\item[SMSModule]
\item[StorageUnit]
\item[StakeHolder]
\item[SystemAdministrator]
\item[Technician]
\item[TelephoneCompany]
\item[ThirdPartyService]
\item[Trame]
\item[UpdateEvent]
\item[UsageDetail]
\item[Utility]
\item[ValveModule]
\item[Water]
\item[WiFiModule]
\end{description}